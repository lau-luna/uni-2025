%%% Template originaly created by Karol Kozioł (mail@karol-koziol.net) and modified for ShareLaTeX use

\documentclass[a4paper,11pt]{article}

\usepackage[T1]{fontenc}
\usepackage[utf8]{inputenc}
\usepackage{graphicx}
\usepackage{subcaption}
\usepackage{xcolor}
\usepackage{stmaryrd}

% \renewcommand\familydefault{\sfdefault}
% \usepackage{tgheros}
% \usepackage[defaultmono]{droidmono}
\usepackage{mathrsfs}

\usepackage{amsmath,amssymb,amsthm,textcomp}
\usepackage{enumerate}
\usepackage{multicol}
\usepackage{tikz}
\usepackage{hyperref}

\usepackage{geometry}
\geometry{left=25mm,right=25mm,%
bindingoffset=0mm, top=20mm,bottom=20mm}

\usepackage[table]{xcolor} % Para colorear tablas
\usepackage{colortbl}      % Necesario para \cellcolor
\usepackage{array}         % Para centrar texto en columnas

\setlength{\parskip}{1em}
%\linespread{1.3}

\newcommand{\linia}{\rule{\linewidth}{0.5pt}}

% % custom theorems if needed
% \newtheoremstyle{mytheor}
%     {1ex}{1ex}{\normalfont}{0pt}{\scshape}{.}{1ex}
%     {{\thmname{#1 }}{\thmnumber{#2}}{\thmnote{ (#3)}}}
%
% \theoremstyle{mytheor}
% \newtheorem{defi}{Definition}

% my own titles
\makeatletter
\renewcommand{\maketitle}{
\begin{center}
\vspace{2ex}
{\Large \textsc{\@title}}
\vspace{1ex}
\\
\linia\\
\@author \hfill \@date
\vspace{4ex}
\end{center}
}
\makeatother
%%%

% custom footers and headers
\usepackage{fancyhdr}
\pagestyle{fancy}
\lhead{}
\chead{}
\rhead{}
%\lfoot{Lógica Proposicional}
\cfoot{}
%\rfoot{página \thepage}
\renewcommand{\headrulewidth}{0pt}
\renewcommand{\footrulewidth}{0pt}
%

% code listing settings
\usepackage{listings}
\lstset{
    language=Python,
    basicstyle=\ttfamily\small,
    aboveskip={1.0\baselineskip},
    belowskip={1.0\baselineskip},
    columns=fixed,
    extendedchars=true,
    breaklines=true,
    tabsize=4,
    prebreak=\raisebox{0ex}[0ex][0ex]{\ensuremath{\hookleftarrow}},
    frame=lines,
    xleftmargin=2em,
    framexleftmargin=2em,
    showtabs=false,
    showspaces=false,
    showstringspaces=false,
    keywordstyle=\color[rgb]{0.627,0.126,0.941},
    commentstyle=\color[rgb]{0.133,0.545,0.133},
    stringstyle=\color[rgb]{01,0,0},
    numbers=left,
    numberstyle=\small,
    stepnumber=1,
    numbersep=5pt,
    captionpos=t,
    escapeinside={\%*}{*)}
}

%% SET THEORY %%

\usepackage{mathtools}
\usepackage{xparse}
\DeclarePairedDelimiterX\set[2]{\{}{\}}{\,#1 \;\delimsize\vert\; #2\,}

\newcommand{\eqdef}{\stackrel{\text{def}}{=}}

%% local labels in equations %%

\usepackage{xargs}

\makeatletter
  \newcommandx{\Label}[1][1={\arabic{equation}}]%
    {\refstepcounter{equation}(\theequation)~\ltx@label{{#1}} &&}
\makeatother

%% Proof trees %%

\usepackage{fitch}

\usepackage{etoolbox}
\let\bbordermatrix\bordermatrix
\patchcmd{\bbordermatrix}{8.75}{4.75}{}{}
\patchcmd{\bbordermatrix}{\left(}{\left[}{}{}
\patchcmd{\bbordermatrix}{\right)}{\right]}{}{}

%% circled question mark %%

\usepackage{tikz}
\newcommand*\circled[1]{\tikz[baseline=(char.base)]{
            \node[shape=circle,draw,inner sep=.7pt] (char) {\footnotesize #1};}}
\newcommand{\result}{\circled{?}}

%% Venn Diagrams %%

\usepackage{venndiagram}

%% LOGICAL SYMBOLS %%

\newcommand{\liff}{\leftrightarrow}
\DeclarePairedDelimiterX\FORALL[3]{(}{)}{\,\forall #1 : #2 : #3 \,}
\DeclarePairedDelimiterX\EXISTS[3]{(}{)}{\,\exists #1 : #2 : #3 \,}
\DeclarePairedDelimiterX\eval[1]{\llbracket}{\rrbracket}{#1}
\newcommand{\sem}[1]{\eval{#1}^{\model}}
\newcommand{\semv}[2]{\eval{#2}^{\model,#1}}
\newcommand{\model}{\mathfrak{M}}
\newcommand{\interpretation}{\mathscr{I}}
\DeclareMathOperator{\dom}{dom}
\DeclareMathOperator{\fun}{fun}
\DeclareMathOperator{\rel}{rel}

\DeclareRobustCommand{\svdots}{% s for `scaling'
  \vcenter{%
    \offinterlineskip
    \vspace{5pt}
    \hbox{.}
    \vskip0.25\normalbaselineskip
    \hbox{.}
    \vskip0.25\normalbaselineskip
    \hbox{.}%
  }%
}

%% COMPLETE BOX %%
\def\fillbox{\quad[\qquad]\quad}

%% NO INDENT %%

\setlength{\parindent}{0pt}

%%%% LOCALDEFS %%%%

\newcommand{\ambassador}{\mathsf{ambassador}_1}
\newcommand{\person}{\mathsf{person}_1}
\newcommand{\country}{\mathsf{country}_1}
\newcommand{\sentto}{\mathsf{sentto}_2}
\newcommand{\knight}{\mathsf{knight}_1}
\newcommand{\knave}{\mathsf{knave}_1}
\newcommand{\alice}{\mathsf{alice}}
\newcommand{\bob}{\mathsf{bob}}
\newcommand{\tom}{\mathsf{tom}}
\newcommand{\carl}{\mathsf{carl}}

%%%----------%%%----------%%%----------%%%----------%%%

\begin{document}

\title{2do Parcial - Lógica y Resolución de Problemas}

\author{Lautaro Luna}

\date{}

\maketitle

\vspace*{-1cm}

Especificaciones y Cálculo Ecuacional

\section{Pregunta (a) [2 puntos]}

Formalizar los predicados:
\begin{enumerate}
	\item $isprime_1(n) \equiv \varphi$
	\item $divides_2(a, b) \equiv \psi$
\end{enumerate}

\subsection{$isprime_1(n) \equiv \varphi$}

$isprime(n) \equiv n > 1 \land (\forall d :: n \mathbin{\%} d = 0 \rightarrow d = 1 \lor d = n)$

\subsection{$divides_2(a, b) \equiv \psi$}

$divides_2(a, b) \equiv b \mathbin{\%} a = 0$

\section{Pregunta (b) [3 puntos]}

\subsection{1. $(\forall i : 0 \leq i < n : isprime_1(a(i)))$}

Es correcta.

Modelo verdadero:

\begin{center}
	\begin{minipage}{0.1 \textwidth}
		\centering
		\textbf{$\bigtriangleup$} \\[4pt]
		\rowcolors{1}{}{blue!80!white}
		\begin{tabular}{>{\columncolor{blue!80!white}\color{white}\centering}m{1em}}
            1 \\
			2 \\
			3 \\
		\end{tabular}
	\end{minipage}
	\begin{minipage}{0.2 \textwidth}
		\centering
		\textbf{$Constantes$} \\[4pt]
		\begin{tabular}{@{}c@{\hskip 1em}>{\columncolor{blue!80!white}\color{white}}c@{}}
			n & 2 \\
		\end{tabular}
	\end{minipage}
	\begin{minipage}{0.2 \textwidth}
		\centering
		\textbf{$a$} \\[4pt]
		\begin{tabular}{c@{\hskip 1em}*{10}{>{\columncolor{blue!80!white}\color{white}}c}} % Aplica solo a columnas de datos
			% Encabezado (no tiene formato blanco)
			\rowcolor{white}
			\multicolumn{1}{c}{}           &
			\multicolumn{1}{c}{\textbf{0}} &
			\multicolumn{1}{c}{\textbf{1}} &
			\\
			% Cuerpo (tiene fondo azul y texto blanco) 
			0                              & 2 & 3 \\
		\end{tabular}
	\end{minipage}

\end{center}

\subsubsection{Demostración}

\begin{nd}
	$
    \hypo{}{\llbracket (\forall i : 0 \leq i < n : isprime_1(a(i))) \rrbracket}
    \hypo{}{\llbracket (\forall i :: 0 \leq i < n \rightarrow isprime_1(a(i))) \rrbracket}
		\have{}{min}
        \open
        \hypo{}{\llbracket 0 \leq i < n \rightarrow isprime_1(a(i)) \rrbracket ~~\text{cuando $[i:=0]$}} 
        
        \open
            \hypo{}{\llbracket 0 \leq i < n \rrbracket }
            \hypo{}{\llbracket ((0 = i) \lor (0 < i)) \land (i < n) \rrbracket}
            \have{}{min}
            \open
                \hypo{}{\llbracket ((0 = i) \lor (0 < i)) \rrbracket}
                \have{}{max}
                \open
                    \hypo{}{(0 = i)}
                    \have{}{1}
                \close
                \have{}{1}
            \close
            \have{}{\llbracket (i < n) \rrbracket} 
            \have{}{1}
        \close

        \open
            \hypo{}{\llbracket isprime_1(a(i)) \rrbracket}
            \hypo{}{\llbracket a(i) > 1 \land (\forall d :: a(i) \mathbin{\%} d = 0 \rightarrow d = 1 \lor d = a(i)) \rrbracket}
            \have{}{min}
            \open
                \hypo{}{a(i) > 1}
                 \have{}{1}
            \close
            \open
                \hypo{}{\llbracket (\forall d :: a(i) \mathbin{\%} d = 0 \rightarrow d = 1 \lor d = a(i)) \rrbracket}
                \have{}{min}
                    \open
                \hypo{}{\llbracket a(i) \mathbin{\%} d = 0 \rightarrow d = 1 \lor d = a(i) \rrbracket ~~\text{cuando [$d = 1$]}}
                \open
                    \hypo{}{\llbracket a(i) \mathbin{\%} d = 0 \rrbracket}
                    \have{}{1}
                \close
                \open
                    \hypo{}{\llbracket d = 1 \lor d = a(i) \rrbracket}
                    \have{}{max}
                    \open
                        \hypo{}{\llbracket d = 1 \rrbracket}
                        \have{}{1}
                    \close
                    \have{}{1}
                \close
                \have{}{1}
            

    $
\end{nd}

\newpage

\begin{nd}
    $
    \open
        \open
            \open
            \open
                \hypo{}{\llbracket a(i) \mathbin{\%} d = 0 \rightarrow d = 1 \lor d = a(i) \rrbracket ~~\text{cuando [$d = 2$]}}
                \open
                    \hypo{}{\llbracket a(i) \mathbin{\%} d = 0 \rrbracket}
                    \have{}{1}
                \close
                \open
                    \hypo{}{\llbracket d = 1 \lor d = a(i) \rrbracket}
                    \have{}{max}
                    \open
                        \hypo{}{\llbracket d = a(i) \rrbracket}
                        \have{}{1}
                    \close
                    \have{}{1}
                \close
                \have{}{1}
            \close     
            \open
                \hypo{}{\llbracket a(i) \mathbin{\%} d = 0 \rightarrow d = 1 \lor d = a(i) \rrbracket ~~\text{cuando [$d = 3$]}}
                \open
                    \hypo{}{\llbracket a(i) \mathbin{\%} d = 0 \rrbracket}
                    \have{}{0}
                \close
                \have{}{1}
            \close

            \have{}{1}

        \close

            \have{}{1}

        \close

            \have{}{1}

        \close
        \open
        \hypo{}{\llbracket 0 \leq i < n \rightarrow isprime_1(a(i)) \rrbracket ~~\text{cuando $[i:=1]$}} 
        
        \open
            \hypo{}{\llbracket 0 \leq i < n \rrbracket \\
                \llbracket ((0 = i) \lor (0 < i)) \land (i < n) \rrbracket}
            \have{}{min}
            \open
                \hypo{}{\llbracket ((0 = i) \lor (0 < i)) \rrbracket}
                \have{}{max}
                \open
                    \hypo{}{(0 < i)}
                    \have{}{1}
                \close
                \have{}{1}
            \close
                \have{}{\llbracket (i < n) \rrbracket} 
            \have{}{1}
        \close


    $
\end{nd}



\newpage

\begin{nd}

    \open
            \open
            \hypo{}{\llbracket isprime_1(a(i)) \rrbracket}
            \hypo{}{\llbracket a(i) > 1 \land (\forall d :: a(i) \mathbin{\%} d = 0 \rightarrow d = 1 \lor d = a(i)) \rrbracket}
            \have{}{min}
            \open
                \hypo{}{a(i) > 1}
                \have{}{1}
            \close
            \open
                \hypo{}{(\forall d :: a(i) \mathbin{\%} d = 0 \rightarrow d = 1 \lor d = a(i))}
                \have{}{min}

                \open
                \hypo{}{\llbracket a(i) \mathbin{\%} d = 0 \rightarrow d = 1 \lor d = a(i) \rrbracket ~~\text{cuando [$d = 1$]}}
                \open
                    \hypo{}{\llbracket a(i) \mathbin{\%} d = 0 \rrbracket}
                    \have{}{1}
                \close
                \open
                    \hypo{}{\llbracket d = 1 \lor d = a(i) \rrbracket}
                    \have{}{max}
                    \open
                        \hypo{}{\llbracket d = 1 \rrbracket}
                        \have{}{1}
                    \close
                    \have{}{1}
                \close
                \have{}{1}
            \close 
            \open
                \hypo{}{\llbracket a(i) \mathbin{\%} d = 0 \rightarrow d = 1 \lor d = a(i) \rrbracket ~~\text{cuando [$d = 2$]}}
                \open
                    \hypo{}{\llbracket a(i) \mathbin{\%} d = 0 \rrbracket}
                    \have{}{0}
                \close
                \have{}{1}
            \close

            \open
                \hypo{}{\llbracket a(i) \mathbin{\%} d = 0 \rightarrow d = 1 \lor d = a(i) \rrbracket ~~\text{cuando [$d = 3$]}}
                \open
                    \hypo{}{\llbracket a(i) \mathbin{\%} d = 0 \rrbracket}
                    \have{}{1}
                \close
                \open
                    \hypo{}{\llbracket d = 1 \lor d = a(i) \rrbracket}
                    \have{}{max}
                    \open
                        \hypo{}{\llbracket d = a(i) \rrbracket}
                        \have{}{1}
                    \close
                    \have{}{1}
                \close
                \have{}{1}
            \close

            \have{}{1}
        \close
        \have{}{1}
    \close
    \have{}{1}
    \close
    \have{}{1}
    $
\end{nd}

$\llbracket (\forall i : 0 \leq i < n : isprime_1(a(i))) \rrbracket = 1$

\newpage

\subsection{2. $(\forall i : 0 \leq i < n : divides_2(1, a(i)) \land divides_2(a(i), a(i))$}

Es incorrecta.

Modelo que debería arrojar falso, pero devuelve verdadero.

\begin{center}
	\begin{minipage}{0.1 \textwidth}
		\centering
		\textbf{$\bigtriangleup$} \\[4pt]
		\rowcolors{1}{}{blue!80!white}
		\begin{tabular}{>{\columncolor{blue!80!white}\color{white}\centering}m{1em}}
			1 \\
			2 \\
			4 \\
		\end{tabular}
	\end{minipage}
	\begin{minipage}{0.2 \textwidth}
		\centering
		\textbf{$Constantes$} \\[4pt]
		\begin{tabular}{@{}c@{\hskip 1em}>{\columncolor{blue!80!white}\color{white}}c@{}}
			n & 1 \\
		\end{tabular}
	\end{minipage}
	\begin{minipage}{0.2 \textwidth}
		\centering
		\textbf{$a$} \\[4pt]
		\begin{tabular}{c@{\hskip 1em}*{10}{>{\columncolor{blue!80!white}\color{white}}c}} % Aplica solo a columnas de datos
			% Encabezado (no tiene formato blanco)
			\rowcolor{white}
			\multicolumn{1}{c}{}           &
			\multicolumn{1}{c}{\textbf{0}} &
			\\
			% Cuerpo (tiene fondo azul y texto blanco) 
			0                              & 4\
		\end{tabular}
	\end{minipage}

\end{center}

\subsubsection{Demostración}

 \begin{nd}
     $
        \hypo{}{\llbracket (\forall i : 0 \leq i < n : divides_2(1, a(i)) \land divides_2(a(i), a(i)) \rrbracket}
        \hypo{}{\llbracket (\forall i :: 0 \leq i < n \rightarrow divides_2(1, a(i)) \land divides_2(a(i), a(i)) \rrbracket}
        \have{}{min}

        \open
        \hypo{}{\llbracket 0 \leq i < n \rightarrow divides_2(1, a(i)) \land divides_2(a(i), a(i)) \rrbracket ~~\text{cuando $[i := 0]$}}
            \open
            \hypo{}{\llbracket 0 \leq i < n \rrbracket \\
                \llbracket ((0 = i) \lor (0 < i)) \land (i < n) \rrbracket}
            \have{}{min}
            \open
                \hypo{}{\llbracket ((0 = i) \lor (0 < i)) \rrbracket}
                \have{}{max}
                \open
                    \hypo{}{(0 = i)}
                    \have{}{1}
                \close
                \have{}{1}
            \close
            \open
                \hypo{}{\llbracket (i < n) \rrbracket} 
                \have{}{1}
            \close
            \have{}{1}
        \close

        \open
            \hypo{}{\llbracket divides_2(1, a(i)) \land divides_2(a(i), a(i)) \rrbracket}
            \have{}{min}
            \open
                \hypo{}{\llbracket divides_2(1, a(i)) \rrbracket}
                \hypo{}{\llbracket a(i) \mathlib{\%} 1 = 0 \rrbracket}
                \have{}{1}
            \close
            \open
                \hypo{}{\llbracket divides_2(a(i), a(i)) \rrbracket}
                \hypo{}{\llbracket a(i) \mathlib{\%} a(i) = 0 \rrbracket}
                \have{}{1}
            \close
            \have{}{1}
        \close
        \have{}{1}
        \close
        \have{}{1}
     $
 \end{nd}

$\llbracket (\forall i : 0 \leq i < n : divides_2(1, a(i)) \land divides_2(a(i), a(i)) \rrbracket = 1$

\newpage

\section{Pregunta (c)}

\subsection{1. }

$(\forall p : ambassador_1(p) : sentto_2(p, france)) \equiv (\forall p : \neg sentto_2(p, france) : \neg ambassador_1(p)) $

\begin{enumerate}
    \item $(\forall p : ambassador_1(p) : sentto_2(p, france))$
    \item $(\forall p :: ambassador_1(p) \rightarrow sentto_2(p, france))$ \hfill (término)
    \item $(\forall p :: \neg sentto_2(p, france) \rightarrow \neg ambassador_1(p))$ \hfill (Teorema 31 de la práctica)
    \item $(\forall p : \neg sentto_2(p, france) : \neg ambassador_1(p))$ \hfill (término) 
\end{enumerate}

$(\exists_p: ambassador_1(p) : sentto_2(p,france)) \equiv (\exists p: sentto_2(p,france) : ambassador_1(p))$\\
1.$(\exists_p: ambassador_1(p) : sentto_2(p,france))$\\
2.$(\exists_p:: ambassador_1(p) \land sentto_2(p,france))$ (Teorema C de lapractica) \\
3.$(\exists_p:: sentto_2(p,france) \land ambassador_1(p) )$ (Teorema 13 de la practica)\\
4.$(\exists_p: sentto_2(p,france) : ambassador_1(p))$ (Teorema 13 de la practica)\\

\end{document}
